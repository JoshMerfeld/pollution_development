% Options for packages loaded elsewhere
\PassOptionsToPackage{unicode}{hyperref}
\PassOptionsToPackage{hyphens}{url}
%
\documentclass[
]{article}
\usepackage{amsmath,amssymb}
\usepackage{lmodern}
\usepackage{iftex}
\ifPDFTeX
  \usepackage[T1]{fontenc}
  \usepackage[utf8]{inputenc}
  \usepackage{textcomp} % provide euro and other symbols
\else % if luatex or xetex
  \usepackage{unicode-math}
  \defaultfontfeatures{Scale=MatchLowercase}
  \defaultfontfeatures[\rmfamily]{Ligatures=TeX,Scale=1}
\fi
% Use upquote if available, for straight quotes in verbatim environments
\IfFileExists{upquote.sty}{\usepackage{upquote}}{}
\IfFileExists{microtype.sty}{% use microtype if available
  \usepackage[]{microtype}
  \UseMicrotypeSet[protrusion]{basicmath} % disable protrusion for tt fonts
}{}
\makeatletter
\@ifundefined{KOMAClassName}{% if non-KOMA class
  \IfFileExists{parskip.sty}{%
    \usepackage{parskip}
  }{% else
    \setlength{\parindent}{0pt}
    \setlength{\parskip}{6pt plus 2pt minus 1pt}}
}{% if KOMA class
  \KOMAoptions{parskip=half}}
\makeatother
\usepackage{xcolor}
\usepackage[margin=1in]{geometry}
\usepackage{longtable,booktabs,array}
\usepackage{calc} % for calculating minipage widths
% Correct order of tables after \paragraph or \subparagraph
\usepackage{etoolbox}
\makeatletter
\patchcmd\longtable{\par}{\if@noskipsec\mbox{}\fi\par}{}{}
\makeatother
% Allow footnotes in longtable head/foot
\IfFileExists{footnotehyper.sty}{\usepackage{footnotehyper}}{\usepackage{footnote}}
\makesavenoteenv{longtable}
\usepackage{graphicx}
\makeatletter
\def\maxwidth{\ifdim\Gin@nat@width>\linewidth\linewidth\else\Gin@nat@width\fi}
\def\maxheight{\ifdim\Gin@nat@height>\textheight\textheight\else\Gin@nat@height\fi}
\makeatother
% Scale images if necessary, so that they will not overflow the page
% margins by default, and it is still possible to overwrite the defaults
% using explicit options in \includegraphics[width, height, ...]{}
\setkeys{Gin}{width=\maxwidth,height=\maxheight,keepaspectratio}
% Set default figure placement to htbp
\makeatletter
\def\fps@figure{htbp}
\makeatother
\setlength{\emergencystretch}{3em} % prevent overfull lines
\providecommand{\tightlist}{%
  \setlength{\itemsep}{0pt}\setlength{\parskip}{0pt}}
\setcounter{secnumdepth}{5}
\newlength{\cslhangindent}
\setlength{\cslhangindent}{1.5em}
\newlength{\csllabelwidth}
\setlength{\csllabelwidth}{3em}
\newlength{\cslentryspacingunit} % times entry-spacing
\setlength{\cslentryspacingunit}{\parskip}
\newenvironment{CSLReferences}[2] % #1 hanging-ident, #2 entry spacing
 {% don't indent paragraphs
  \setlength{\parindent}{0pt}
  % turn on hanging indent if param 1 is 1
  \ifodd #1
  \let\oldpar\par
  \def\par{\hangindent=\cslhangindent\oldpar}
  \fi
  % set entry spacing
  \setlength{\parskip}{#2\cslentryspacingunit}
 }%
 {}
\usepackage{calc}
\newcommand{\CSLBlock}[1]{#1\hfill\break}
\newcommand{\CSLLeftMargin}[1]{\parbox[t]{\csllabelwidth}{#1}}
\newcommand{\CSLRightInline}[1]{\parbox[t]{\linewidth - \csllabelwidth}{#1}\break}
\newcommand{\CSLIndent}[1]{\hspace{\cslhangindent}#1}
\newcommand{\beginappendix}{ \setcounter{table}{0} \renewcommand{\thetable}{A\arabic{table}} \setcounter{figure}{0} \renewcommand{\thefigure}{A\arabic{figure}} }
\usepackage[capposition=top]{floatrow}
\usepackage{placeins}
\usepackage{setspace}
\usepackage{dcolumn}
\usepackage{booktabs}
\usepackage{siunitx}
\usepackage{amsmath}
\usepackage{enumerate}
\usepackage[shortlabels]{enumitem}
\usepackage[hang,flushmargin]{footmisc}
\usepackage{booktabs}
\usepackage{longtable}
\usepackage{array}
\usepackage{multirow}
\usepackage{wrapfig}
\usepackage{float}
\usepackage{colortbl}
\usepackage{pdflscape}
\usepackage{tabu}
\usepackage{threeparttable}
\usepackage{threeparttablex}
\usepackage[normalem]{ulem}
\usepackage{makecell}
\usepackage{xcolor}
\ifLuaTeX
  \usepackage{selnolig}  % disable illegal ligatures
\fi
\IfFileExists{bookmark.sty}{\usepackage{bookmark}}{\usepackage{hyperref}}
\IfFileExists{xurl.sty}{\usepackage{xurl}}{} % add URL line breaks if available
\urlstyle{same} % disable monospaced font for URLs
\hypersetup{
  pdftitle={Pollution, agricultural productivity, and development: Evidence from coal in plants in India},
  pdfauthor={Joshua D. Merfeld},
  hidelinks,
  pdfcreator={LaTeX via pandoc}}

\title{Pollution, agricultural productivity, and development: Evidence from coal in plants in India\footnote{thanks to\ldots{}}}
\author{Joshua D. Merfeld\footnote{KDI School of Public Policy and Management and IZA; \href{mailto:merfeld@kdis.ac.kr}{\nolinkurl{merfeld@kdis.ac.kr}}}}
\date{2023-01-09}

\begin{document}
\maketitle
\begin{abstract}
\noindent abstract\\
\strut \\
\textbf{\textit{Keywords}}:\\
\textbf{\textit{JEL Codes}}:
\end{abstract}

\newpage
\doublespacing

\hypertarget{introduction}{%
\section{Introduction}\label{introduction}}

pollution could be bad for agriculture (Heck et al. 1982; Marshall et al. 1997)

lower agricultural productivity near gold mines in Ghana (Aragón and Rud 2016); apparently lots of pollution
- ``The main identification assumption is that the change in agricultural productivity over time in both areas would be similar in the absence of mining.''
- within 20km
- also finds suggestive evidence it's not all labor productivity changes

``Much of what we know about the marginal effect of pollution on infant mortality is derived from developed country data. However, given the lower levels of air pollution in developed countries, these estimates may not be externally valid to the developing country context if there is a non-linear dose relationship between pollution and mortality or if the costs of avoidance behaviour differ considerably between the two contexts. In this article, we estimate the relationship between pollution and infant mortality using data from Mexico. Our estimates for PM10 tend to be similar (or even smaller) than the US estimates, while our findings on CO tend to be larger than those derived from the US context.'' (Arceo, Hanna, and Oliva 2016)

We have known about the effects of pollution on human health for many years (Brunekreef and Holgate 2002; Kampa and Castanas 2008; Pope III and Dockery 2006)
good review on the effects of pollution early in life in Currie et al. (2014)
- also in developing countries, where there is a consistent effect of exposure on infant mortality (Heft-Neal et al. 2018, 2019)

``Air pollution has both acute and chronic effects on human health, affecting a number of different systems and organs. It ranges from minor upper respiratory irritation to chronic respiratory and heart disease, lung cancer, acute respiratory infections in children and chronic bronchitis in adults, aggravating pre-existing heart and lung disease, or asthmatic attacks. In addition, short- and long-term exposures have also been linked with premature mortality and reduced life expectancy. These effects of air pollutants on human health and their mechanism of action are briefly discussed.'' Kampa and Castanas (2008)

worker productivity and cognitive function:
- call center workers in China, and at common levels of pollution (Chang et al. 2019)
- tests (Ebenstein, Lavy, and Roth 2016; Wen and Burke 2022)
- farm workers in California (Graff Zivin and Neidell 2012)
- increase in extensive margin (Hanna and Oliva 2015)
- manufacturing in China (He, Liu, and Salvo 2019); ``We uncover statistically significant adverse output effects from more prolonged exposure, but effects are not large. A substantial +10 mug/m3 PM2.5 variation sustained over 25 days reduces daily output by 1 percent.''

induces migration (Chen, Oliva, and Zhang 2022)

overall effects on agricultural productivity taking into accounts any changes in input allocation caused by the increase in pollution.

\hypertarget{data-and-methods}{%
\section{Data and methods}\label{data-and-methods}}

\hypertarget{data}{%
\subsection{Data}\label{data}}

The main goal of this paper is to examine whether exposure to exhaust from coal plants affects agricultural productivity, labor allocation, and economic growth. I use several sources of data, which I describe here and in Table \ref{tab:data} in the appendix.

The first set of data lists the location of coal plant across the globe. This data comes from Global Energy Monitor.\footnote{\url{https://globalenergymonitor.org/projects/global-coal-plant-tracker/}} and lists all units generating at least 30 megawatts of electricity. The data on coal plants includes the year of opening (and, if applicable, the year of retirement), the GPS (latitude/longitude) location of each plant, and the amount of power produced by the units at each plant. For this paper, I do not use the information on the capacity of the plant.

\begin{figure}
\includegraphics{draft_files/figure-latex/plants-1} \caption[Coal plants in India from 1990 to 2010]{Coal plants in India from 1990 to 2010}\label{fig:plants}\floatfoot*{Note: The top figure shows the location of coal plants in 1990. The bottom figure shows the location of coal plants in 2010.}
\end{figure}

Figure \ref{fig:plants} presents the location of coal plants in India for two specific years: 1990 and 2010. There is a clear increase in the prevalence of coal plants across the country over the two decades. Additionally, the overall capacity from coal plants increased from 42.4 gigawatts to 100.4 gigawats, an increase of 136.6 percent in just 20 years.

The second dataset includes agricultural productivity for both the monsoon and winter seasons, from 2002 to 2018. To match other data, I only use the data up to 2013. This data comes from Gangopadhyay et al. (2022) and estimates land productivity (i.e.~yield, in tons per hectare) for the major crops in India. They define the monsoon season as June to October and the winter season as November to March. I keep these definitions when matching across data below. This data is also publicly available from Nature's data-sharing website.\footnote{\url{https://springernature.figshare.com}}

The agricultural productivity data is available as raster files with a resolution of 500m. To aggregate this data up to a useful administrative unit, I use village-level shapefiles provided by Asher et al. (2021) and publicly available on the SHRUG platform.\footnote{\url{https://www.devdatalab.org/shrug_download/}} I aggregate the agricultural productivity data to the village level by extracting mean productivity for each feature in the shapefile. I do this separately for each season -- monsoon and winter -- and each year, from 2002 to 2013.

To measure exposure to pollution from coal plants, I first locate all village centroids located within 30km from a coal plant in a given year. I choose 30km due to previous research on the effects of (air) pollution (Aragón and Rud 2016; Bencko and Symon 1977; Li and Gibson 2014). I then calculate the direction from coal plants to all village centroids within that 30km radius. To define exposure, I then pull daily wind direction data from the National Center for Atmospheric Research.\footnote{\url{https://climatedataguide.ucar.edu/}} For each day, I document whether the wind is blowing towards each village centroid.\footnote{I define ``towards'' as within five degrees to help take into account that I am using village centroids, using the x and y components of wind speed.} I then temporally aggregate this daily data depending on the temporal definition of the corresponding outcome. For example, agricultural productivity is defined across five months (e.g.~the monsoon season is from June to October) so I count the total days a given village is exposed to wind during those five months.

\begin{figure}
\includegraphics{draft_files/figure-latex/windexample-1} \caption[Wind direction and aggregation examples (2010-01-01)]{Wind direction and aggregation examples (2010-01-01)}\label{fig:windexample}\floatfoot*{Note: The left figure shows the average wind direction on January 1st, 2010. The points are the location of coal plants on that date. The right figure shows the distribution of pollution exposure in a specific district -- Kangra district in Himachal Pradesh -- on the same date.}
\end{figure}

The top panel in Figure \ref{fig:windexample} is an example wind direction raster. The raster shows wind direction for the entirety of India on January 1st, 2010, as well as the location of coal power plants on that date. The prevailing winds on the date differ across the country and, though not shown in the figure, across days. This means that the overall exposure for a given area to coal plant emissions changes across time. I also pull data on particulate matter to help validate this proxy for exposure. This data comes from Hammer et al. (2020) and I also aggregate this to the village using the same method detailed above.

To help understand some of the mechanisms driving the relationship between exposure to emissions and agricultural productivity, I also include household survey data. Unfortunately, I am not aware of any publicly available survey data that covers multiple years and has identifiers below the district level. For example, the ARIS-REDS data\footnote{\url{https://www.ncaer.org/project/additional-rural-incomes-survey-aris-rural-economic-demographic-survey-reds}} includes GPS identifiers for one year, but that only allows GPS matching for a total of two years: 1999 and 2007/2008. The DHS, on the other hand, only has district-level identifiers (except for the newest two waves) and is also not a labor survey. Instead, I use the National Sample Survey (NSS) from the Ministry of Statistics and Programme Implementation.

The NSS is a nationally representative survey that was conducted every couple of years until relatively recently. Many rounds of the NSS also included a module on employment that asked about labor allocation over the previous seven days. I use the 61st, 62nd, 64th, 66th, and 68th rounds of the NSS, all of which include the employment module. The publicly available data also includes the date of survey, which allows me to match pollution exposure to the exact day of the survey.

The biggest issue with using the NSS is that it includes only district-level identifiers. The main exposure variables, on the other hand, are defined at a lower level of aggregation, at the village. I choose to aggregate this village-level exposure to the district level as shown in the second panel of Figure \ref{fig:windexample}. I use the area overlap of the villages with the district to define a weighted average of exposure for the district at a given time. I calculate the area of overlap for each village and the district as well as the overall area of the district and use these variables to define weights for individual villages. Any part of the district that does not overlap with one of the villages -- the white area in Figure \ref{fig:windexample} -- does not have any exposure and, as such, receives a value of zero for exposure.

Finally, I use nightlights as a measure of economic growth (Henderson, Storeygard, and Weil 2012). I also take this data directly from SHRUG and Asher et al. (2021). This covers 1992 through 2013. Some of the older data may not be ideal (Gibson et al. 2021), but this is the only way to document changes in economic growth at a more disaggregated level and at a regular basis. Although satellite imagery is an attractive alternative (Burke et al. 2021), high-resolution satellite imagery does not go back to the early 1990s.

\hypertarget{identification}{%
\subsection{Identification}\label{identification}}

To calculate the effects of exposure to pollution on different outcomes, I rely on fixed effects and the differential exposure driven by plausibly exogenous wind direction, conditional on total rainfall. The regession is of the form
\begin{gather} y_{it} = \alpha_{i} + \gamma_{t} + \beta wind_{it} + \delta rain_{it} + \varepsilon_{it}, \end{gather}
where \(y_{it}\) is outcome \(y\) for unit \(i\) at time \(t\), \(\alpha\) is a geographic fixed effects, \(\gamma_t\) is time fixed effects, \(wind_{it}\) is exposure to pollution -- defined using wind direction -- and \(rain_{it}\) is rainfall. Although I use \(i\) and \(t\) as subscripts for the fixed effects, note that these change depending on the unit of analysis and the outcome. I explicitly state the level of fixed effects in the tables and results below.

Identification relies on changes in wind direction across time for the same geographic units. In other words, conditional on the fixed effects and rainfall, I assume changes in wind direction are as good as random and uncorrelated with the outcome -- e.g.~agricultural productivity -- except through exposure to pollution. One major threat to identification is that the building of a coal plant could lead to a spurious correlation between the exposure variable and the outcome. Specifically, since exposure is defined as zero prior to the building of a coal plant, the exposure variable itself might be endogenous even though wind direction is not. To help allay these concerns, I estimate additional regressions restricting the sample to only time periods after which a given location has a coal plant. I also present results using leads to show that future wind direction does not seem to predict current outcomes.

\hypertarget{an-example-validating-wind-direction}{%
\subsection{An example: validating wind direction}\label{an-example-validating-wind-direction}}

Before moving on to the main results, it is worth taking the time to better understand the fixed effects strategy with an example. Consider the data used for the regressions presented in Table \ref{tab:pollutiontable}. The outcome variable is particulate matter -- specifically, \(\mathrm{PM_{2.5}}\), which is particulate matter no larger than 2.5 micrometers in diameter -- which comes from Hammer et al. (2020). \(\mathrm{PM_{2.5}}\) is one of the harmful byproducts from coal plants, along with sulfur dioxide, different types of nitrogen oxides, and mercury.\footnote{\url{https://www.epa.gov/airmarkets/power-plants-and-neighboring-communities}} This data is available at the monthly level, so in Table \ref{tab:pollutiontable}, I aggregate wind exposure to the monthly level, as well.

\begin{table}

\caption{\label{tab:pollutiontable}Wind direction and particulate matter}
\centering
\begin{threeparttable}
\begin{tabular}[t]{>{\raggedright\arraybackslash}p{4cm}>{\centering\arraybackslash}p{2cm}>{\centering\arraybackslash}p{2cm}>{\centering\arraybackslash}p{2cm}>{\centering\arraybackslash}p{2cm}}
\toprule
\multicolumn{1}{c}{ } & \multicolumn{2}{c}{1998-2015} & \multicolumn{2}{c}{2002-2013} \\
\cmidrule(l{3pt}r{3pt}){2-3} \cmidrule(l{3pt}r{3pt}){4-5}
  & (1) & (2) & (3) & (4)\\
\midrule
wind & 0.045*** & 0.014*** & 0.063*** & 0.015***\\
 & (0.004) & (0.001) & (0.005) & (0.002)\\
\textbf{fixed effects:} & \textbf{} & \textbf{} & \textbf{} & \textbf{}\\
village & Yes & Yes & Yes & Yes\\
month & Yes & No & Yes & No\\
district-month & No & Yes & No & Yes\\
\midrule
observations & 22,345,092 & 22,345,092 & 14,896,728 & 14,896,728\\
\bottomrule
\end{tabular}
\begin{tablenotes}
\item Note: Standard errors are in parentheses and are clustered at the village level.
\item * p<0.10 ** p<0.05 *** p<0.01
\end{tablenotes}
\end{threeparttable}
\end{table}

Both the particulate matter data and the wind data is at the village level, which means the regressions in Table \ref{tab:pollutiontable} are at the village level. As such, the geographic fixed effects are village fixed effects and the temporal fixed effects are month fixed effects. Since the exposure variable is at the village level and we follow villages over time, the standard errors are also clustered at the village level.

In addition to serving as an example, Table \ref{tab:pollutiontable} also helps validate the use of wind direction as a proxy for exposure to pollution. The first two columns include all available years of data while the last two columns restrict estimation only to the years for which the agricultural productivity regressions are estimated. Across all four columns, the story is the same: when the wind is blowing in the direction of a village from a coal plant within 30 kilometers, estimated \(\mathrm{PM_{2.5}}\) is substantially higher. The regressions indicate that one additional day of wind in the direction of the village increases mean \(\mathrm{PM_{2.5}}\) for the month by between 0.014 and 0.045 \(\mathrm{\mu g/m^3}\).

To put these numbers into perspective, the World Health Organization's updated guidelines are that average annual exposure should not exceed 15 \(\mathrm{\mu g/m^3}\). For the month, an increase at the midpoint of the estimated range is equal to an increase of 0.2 percent of the maximum recommended mean concentration from the WHO. This is just a single day of wind in the direction of a village. Since this includes all villages located within 30km of a coal plant, this is evidence that wind direction can have serious repercussions on the health of hundreds of millions of people in India.

\hypertarget{results}{%
\section{Results}\label{results}}

This section presents the main results of this paper. I present three sets of results: one set looking at agricultural productivity, another looking at labor allocation, and a final set looking at nightlights, which are a proxy for economic development. I go through each of these in turn.

\hypertarget{agricultural-productivity}{%
\subsection{Agricultural productivity}\label{agricultural-productivity}}

I first present results for agricultural productivity. The outcome of all the regressions in this section is the log of agricultural land productivity, which is defined as tons per hectare. The first set of results are in Table \ref{tab:yieldtable}. The unit of analysis is the year season -- from 2002 to 2013 -- the geographic fixed effect is the village-season, and the temporal fixed effect is the year. The exposure variable is defined for the entire season. Since Gangopadhyay et al. (2022) define both the monsoon (kharif) season and the winter (rabi) season as five months long, I do as well. This implies that the exposure variable ranges from zero to slightly more than 150.

The first column presents the most simple results, with only the wind exposure variable and the fixed effects. The second column adds a rainfall variable -- defined as the total rainfall in the season -- the addition of which is motivated by the fact that changes in wind direction could be correlated with changes in weather. However, the coefficient on the wind exposure variable is completely unchanged by the addition of rainfall, indicating that weather may not be as worrisome of a confounder as thought.

Column three adds district-by-year-by-season fixed effects as an additional robustness checks. This addition decreases the magnitude of the exposure variable but it remains negative and significant. However, part of this may be due to the fact that the new fixed effects soak up much of the variation in the exposure variable, as shown in Figure \ref{fig:windexample}. The last two columns separate the effects into the monsoon season and the winter season, but the overall effect is roughly similar, though rainfall has a larger effect in the monsoon season than the winter season, which is unsurprising.

\begin{table}

\caption{\label{tab:yieldtable}Wind direction and agricultural productivity}
\centering
\begin{threeparttable}
\begin{tabular}[t]{>{\raggedright\arraybackslash}p{3cm}>{\centering\arraybackslash}p{2cm}>{\centering\arraybackslash}p{2cm}>{\centering\arraybackslash}p{2cm}>{\centering\arraybackslash}p{2cm}>{\centering\arraybackslash}p{2cm}}
\toprule
\multicolumn{1}{c}{ } & \multicolumn{3}{c}{all} & \multicolumn{1}{c}{monsoon} & \multicolumn{1}{c}{winter} \\
\cmidrule(l{3pt}r{3pt}){2-4} \cmidrule(l{3pt}r{3pt}){5-5} \cmidrule(l{3pt}r{3pt}){6-6}
  & (1) & (2) & (3) & (4) & (5)\\
\midrule
wind & -0.003*** & -0.003*** & -0.0007*** & -0.002*** & -0.003***\\
 & (0.0002) & (0.0002) & (9.21e-5) & (0.0002) & (0.0003)\\
rain (z) &  & 0.029*** & 0.009*** & 0.082*** & 0.016***\\
 &  & (0.0004) & (0.001) & (0.002) & (0.0004)\\
\textbf{fixed effects:} & \textbf{} & \textbf{} & \textbf{} & \textbf{} & \textbf{}\\
village-season & Yes & Yes & Yes & No & No\\
year & Yes & Yes & No & Yes & Yes\\
district-year-season & No & No & Yes & No & No\\
village & No & No & No & Yes & Yes\\
\midrule
observations & 2,391,533 & 2,375,337 & 2,375,337 & 1,259,123 & 1,116,214\\
\bottomrule
\end{tabular}
\begin{tablenotes}
\item Note: Standard errors are in parentheses and are clustered at the village level.
\item * p<0.10 ** p<0.05 *** p<0.01
\end{tablenotes}
\end{threeparttable}
\end{table}

To put the size of the coefficient in context, it is worth digging a little deeper into the exposure variable. The mean within-village absolute deviation in wind exposure is approximately 8.06 days, meaning that the average change in agricultural productivity from year-to-year due to nothing but changes in wind patterns carrying particulate matter is around 2.4 percent relative to the mean. This implies that swings in agricultural productivity -- due to wind direction -- of more than five percent are probably quite common, since the absolute deviation includes deviations below and above the mean.

\begin{table}

\caption{\label{tab:yieldtabletwo}Pollution and agricultural productivity, IV estimates}
\centering
\begin{threeparttable}
\begin{tabular}[t]{>{\raggedright\arraybackslash}p{3.5cm}>{\centering\arraybackslash}p{2cm}>{\centering\arraybackslash}p{2cm}>{\centering\arraybackslash}p{2cm}>{\centering\arraybackslash}p{2cm}>{\centering\arraybackslash}p{2cm}}
\toprule
\multicolumn{1}{c}{ } & \multicolumn{3}{c}{all} & \multicolumn{1}{c}{monsoon} & \multicolumn{1}{c}{winter} \\
\cmidrule(l{3pt}r{3pt}){2-4} \cmidrule(l{3pt}r{3pt}){5-5} \cmidrule(l{3pt}r{3pt}){6-6}
  & (1) & (2) & (3) & (4) & (5)\\
\midrule
particulate matter & -0.021*** & -0.020*** & -0.033*** & -0.013*** & -0.024***\\
(PM 2.5) & (0.001) & (0.002) & (0.005) & (0.001) & (0.003)\\
rain (z) &  & 0.004** & 0.002 & 0.086*** & -0.016***\\
 &  & (0.002) & (0.002) & (0.002) & (0.004)\\
\textbf{fixed effects:} & \textbf{} & \textbf{} & \textbf{} & \textbf{} & \textbf{}\\
village-season & Yes & Yes & Yes & No & No\\
year & Yes & Yes & No & Yes & Yes\\
district-year-season & No & No & Yes & No & No\\
village & No & No & No & Yes & Yes\\
\midrule
observations & 2,391,533 & 2,375,337 & 2,375,337 & 1,259,123 & 1,116,214\\
\midrule
\textbf{first stage:} & \textbf{} & \textbf{} & \textbf{} & \textbf{} & \textbf{}\\
wind & 0.143*** & 0.126*** & 0.022*** & 0.155*** & 0.105***\\
 & (0.003) & (0.003) & (0.002) & (0.003) & (0.004)\\
rain (z) &  & -1.23*** & -0.235*** & 0.301*** & -1.36***\\
 &  & (0.010) & (0.018) & (0.015) & (0.009)\\
\bottomrule
\end{tabular}
\begin{tablenotes}
\item Note: Standard errors are in parentheses and are clustered at the village level.
\item * p<0.10 ** p<0.05 *** p<0.01
\end{tablenotes}
\end{threeparttable}
\end{table}

If you believe that wind direction is exogenous and, conditional on the fixed effects and rainfall, only affects agricultural productivity through pollution -- which seems reasonable since it is hard to imagine what other variables it might affect -- then it is a valid instrument for
pollution. While I opt to present the reduced form results for the rest of this paper, it is nonetheless an interesting exercise to also use wind direction as an instrument for pollution, which I measure using estimated \(\mathrm{PM_{2.5}}\) from Hammer et al. (2020).

The IV results appear in Table \ref{tab:yieldtabletwo}, with the second-stage results in the top panel and the first-stage results in the bottom panel. The first-stage results make clear how strong of an instrument wind is; also note that the coefficients are quite different from those in Table \ref{tab:pollutiontable} because the unit of analysis in Table \ref{tab:yieldtabletwo} is the season, while in Table \ref{tab:pollutiontable} the unit of analysis is the month.

The second-stage results indicate that \(\mathrm{PM_{2.5}}\) is a very strong predictor of agricultural productivity. The smallest coefficient -- that for the monsoon season in column (4) -- and the mean absolute deviation for \(\mathrm{PM_{2.5}}\) point to a change in agricultural productivity of around 23 percent.

\begin{table}

\caption{\label{tab:yieldtablehet}Heterogeneity in the effects of pollution on productivity}
\centering
\begin{threeparttable}
\begin{tabular}[t]{>{\raggedright\arraybackslash}p{4cm}>{\centering\arraybackslash}p{2cm}>{\centering\arraybackslash}p{2cm}>{\centering\arraybackslash}p{2cm}>{\centering\arraybackslash}p{2cm}>{\centering\arraybackslash}p{2cm}}
\toprule
\multicolumn{1}{c}{ } & \multicolumn{1}{c}{>p(50)} & \multicolumn{1}{c}{<=p(50)} & \multicolumn{3}{c}{all} \\
\cmidrule(l{3pt}r{3pt}){2-2} \cmidrule(l{3pt}r{3pt}){3-3} \cmidrule(l{3pt}r{3pt}){4-6}
  & (1) & (2) & (3) & (4) & (5)\\
\midrule
wind & -0.003*** & -0.032*** & -0.004*** & 0.0004 & -0.0001\\
 & (0.0002) & (0.002) & (0.0003) & (0.0005) & (0.001)\\
rain (z) & 0.031*** & 0.030*** & 0.029*** & 0.029*** & 0.029***\\
 & (0.0006) & (0.0005) & (0.0004) & (0.0004) & (0.0004)\\
wind x rain &  &  & 0.0005*** &  & \\
 &  &  & (6.67e-5) &  & \\
wind squared &  &  &  & -0.0002*** & \\
 &  &  &  & (3.54e-5) & \\
wind x starting yield &  &  &  &  & -0.001**\\
 &  &  &  &  & (0.0006)\\
\textbf{fixed effects:} & \textbf{} & \textbf{} & \textbf{} & \textbf{} & \textbf{}\\
village-season & Yes & Yes & Yes & Yes & Yes\\
year & Yes & Yes & Yes & Yes & Yes\\
\midrule
observations & 1,115,694 & 1,259,643 & 2,375,337 & 2,375,337 & 2,371,364\\
\bottomrule
\end{tabular}
\begin{tablenotes}
\item Note: Standard errors are in parentheses and are clustered at the village level.
\item * p<0.10 ** p<0.05 *** p<0.01
\end{tablenotes}
\end{threeparttable}
\end{table}

Having established that exposure to pollution has a negative effect on agricultural productivity, I now move to heterogeneity in these effects. Table \ref{tab:yieldtablehet} presents four different analyses: effects of exposure by maximum exposure, multiplicative effects of shocks, the possibility of non-linearities, and the differential effects of exposure based on initial agricultural productivity.

The first two columns split the sample based on average exposure; villages with maximum exposure above the median are included in column (1) while those below the median are in column (2). Interestingly, the largest effects are seen on villages in the lower half of the distribution. Apparently, smaller maximum effects lead to larger overall effects of pollution exposure on agricultural productivity. In other words, it is not the places where wind is most likely to blow from a coal plant that experience the largest decrease in productivity in response to an additional day of wind-driven pollution.

Column three looks add the possible multiplicative effects of rainfall shocks and pollution shocks on agricultural productivity. In previous tables, rainfall always entered the regression positively -- as we would expect -- indicating that higher rainfall in a given season leads to higher agricultural productivity. However, an interesting question is whether rainfall and pollution shocks might interact. The positive coefficient on the interaction term between days of wind and rainfall during the agricultural season indicates that there is a multiplicative effect of shocks; when wind days are higher (bad) and rainfall is lower (bad), the overall negative effects are even worse than when just one shock happens. Considering that climate change is leading to more variable rainfall and more negative rainfall shocks, these results are potentially worrisome, especially for smallholders.

Column four adds a quadratic in wind to see if the negative effects of pollution on agricultural productivity are increase or decreasing in wind. In column four, the coefficient on the linear term is 0.0004 while the coefficient on the quadratic term is -0.0002, meaning that the effect of wind is decreasing for any days of wind over two. The negative coefficient also indicates that this negative effect is consistently decreasing, meaning the effect gets worse as average wind days gets higher. Note that this is somewhat different from the regressions in the first two columns, which split the sample by the \emph{maximum} days of wind observed in the sample, not the mean.

In the final column I add to the regression an interaction between wind and initial agricultural productivity (defined in 2002, the first year observed in the productivity data). The linear term drops out of the regression since it is invariant within villages, leaving just the interaction term. The negative coefficient on the interaction term of -0.001 indicates that the effect of wind is more negative in areas where agricultural productivity is higher. Note that agricultural productivity is defined in logs, so this can be interpreted as a percentage effect, not a level effect. Apparently, wind-driven pollution is particularly detrimental to agricultural productivity in the most productive places.

\hypertarget{robustness-checks}{%
\subsection{Robustness checks}\label{robustness-checks}}

I present two separate robustness checks in the appendix. The first set of analyses attempts to separate the possible endogenous location of coal plants from outcomes. Specifically, prior to the opening of a new coal plant in an area, all villages receive a zero for wind. This means that, even if wind direction is exogenous, the opening of the coal plant could lead to correlations between the error term and the wind variable.

Although the results with district-year-season help mitigate this concern, Table \ref{tab:yieldtablepostplant} in the appendix checks this possibility in a different way. Specifically, I restrict the sample to only village-years \emph{after} the introduction of a coal plant. This drops any pre-plant-opening observations -- all of which are mechanically zeros for wind -- to prevent endogeneity due to the opening. The coefficients in Table \ref{tab:yieldtablepostplant} show that the overall effects are essentially unchanged, with just a small change in coefficient magnitude (driven mostly by rounding).

The second robustness check includes leads for wind to see if they predict current year productivity. I present these results in \ref{tab:yieldtableleads}. The second column includes district-year-season fixed effects, while the first does not. The results make clear the importance of including trends or district-year fixed effects. In the first column, leads seem to predict productivity. However, this relationship disappears in the second column, indicating that the previous results with district-year fixed effects are likely the most reliable.

\hypertarget{labor-allocation}{%
\subsection{Labor allocation}\label{labor-allocation}}

The previous set of results show a consistent negative effect of wind-driven pollution on agricultural productivity. A key question is the mechanism through which this happens; is this driven by changes in labor allocation/productivity or by changes in land productivity? While I cannot definitely prove the mechanism, I can shed some light on possibilities by looking at labor allocation decisions. Since these shocks generally happen after initial planting decisions, it seems unlikely that changes in land \emph{allocation} could drive the results.

Table \ref{tab:labortable} presents results for labor allocation over the last seven days using the National Sample Survey and the methodology described above. Since the labor recall is for seven days, labor allocation ranges from zero to seven, as does the wind variable. Recall that the data is at the district level, not the village level. However, it is also more temporally disaggregated, with rolling interviews throughout the year-long survey. Figure \ref{fig:laborplot} in the appendix shows this temporal distribution across the five waves of the survey.

There are some interesting results in Table \ref{tab:labortable}. The first two columns show a marginally significant decrease in overall labor allocation (where ``labor'' is defined as non-domestic self and wage employment). The third and fourth columns show that this overall decrease is predominantly from wage employment, not self employment. An additional day of wind over the last seven decreases wage employment by 0.034 days in the district. Mean wage employment is 1.458 days, meaning one day of wind decreases wage employment by 2.3 percent of the mean. Since much of the district is not covered by ``treated'' villages, this is actually a rather substantive effect on the villages involved.

The last two columns split the effects by sector, into farm and non-farm employment. Although agricultural (land) productivity decreases when wind blows in the direction of villages, overall labor allocation to farm labor actually goes \emph{up}, by around 4.5 percent of the mean of 0.784 days. Non-farm labor, on the other hand, goes down, by around 2.4 percent of the mean of 2.527 days.

There are two possible explanations for this pattern of results. First, it could be that land productivity goes down, leading farmers to increase their farm labor allocation in an attempt to compensate for the decreased productivity. A second possibility is that pollution does not affect land productivity at all, at least not directly. In this scenario, the worsening air quality could lead to decreased labor productivity, meaning that ``effective labor'' essentially decreases when pollution is higher. Unfortunately, I am not able to disentangle these two possiblities with the data available.

Table \ref{tab:labortablemonsoon} and Table \ref{tab:labortablewinter} in the appendix disaggregate these effects into the monsoon and winter seasons. Apparently, the entirety of the effect is driven by changes in labor allocation during the monsoon season. The dynamics across seasons apparently lead to large differences in the effects of pollution on labor allocation.

\begin{table}

\caption{\label{tab:labortable}Wind direction and labor allocation}
\centering
\begin{threeparttable}
\begin{tabular}[t]{>{\raggedright\arraybackslash}p{3cm}>{\centering\arraybackslash}p{1.5cm}>{\centering\arraybackslash}p{1.5cm}>{\centering\arraybackslash}p{1.5cm}>{\centering\arraybackslash}p{1.5cm}>{\centering\arraybackslash}p{1.5cm}>{\centering\arraybackslash}p{1.5cm}}
\toprule
  & all & all & self & wage & farm & non-farm\\
\midrule
wind & -0.021 & -0.027* & 0.008 & -0.034** & 0.035* & -0.061**\\
 & (0.015) & (0.015) & (0.017) & (0.016) & (0.019) & (0.025)\\
controls & No & Yes & Yes & Yes & Yes & Yes\\
\textbf{fixed effects:} & \textbf{} & \textbf{} & \textbf{} & \textbf{} & \textbf{} & \textbf{}\\
district & Yes & Yes & Yes & Yes & Yes & Yes\\
year & Yes & Yes & Yes & Yes & Yes & Yes\\
\textbf{varying slopes:} & \textbf{} & \textbf{} & \textbf{} & \textbf{} & \textbf{} & \textbf{}\\
year (by district) & Yes & Yes & Yes & Yes & Yes & Yes\\
\midrule
observations & 899,045 & 898,856 & 898,856 & 898,856 & 898,856 & 898,856\\
\bottomrule
\end{tabular}
\begin{tablenotes}
\item Note: Standard errors are in parentheses and are clustered at the district level. Control variables include female, age, age squared, and (years of) education.
\item * p<0.10 ** p<0.05 *** p<0.01
\end{tablenotes}
\end{threeparttable}
\end{table}

\hypertarget{nightlights}{%
\subsection{Nightlights}\label{nightlights}}

Pollution leads to lower agricultural productivity as well as changes in labor allocation. This leads to an obvious question: might higher levels of pollution also directly affect economic growth? To explore this question, I look at changes in nightlights relative to changes in pollution. It seems reasonable to think that pollution might affect nightlights with a lag, so I include the lag of pollution rather than contempraneous pollution.

\begin{table}

\caption{\label{tab:ntltable}Wind direction and nightlight growth}
\centering
\begin{threeparttable}
\begin{tabular}[t]{>{\raggedright\arraybackslash}p{3cm}>{\centering\arraybackslash}p{2cm}>{\centering\arraybackslash}p{2cm}}
\toprule
  & (1) & (2)\\
\midrule
wind (lagged) & -0.080*** & -0.083***\\
 & (0.015) & (0.015)\\
rain (z, lagged) &  & -0.035***\\
 &  & (0.002)\\
\textbf{fixed effects:} & \textbf{} & \textbf{}\\
village & Yes & Yes\\
year & Yes & Yes\\
\textbf{varying slopes:} & \textbf{} & \textbf{}\\
year (by village) & Yes & Yes\\
\midrule
observations & 2,146,715 & 2,146,715\\
\bottomrule
\end{tabular}
\begin{tablenotes}
\item Note: Standard errors are in parentheses and are clustered at the village level.
\item * p<0.10 ** p<0.05 *** p<0.01
\end{tablenotes}
\end{threeparttable}
\end{table}

I present these results in Table \ref{tab:ntltable}. To help simplify the coefficients, I recode wind to be the proportion of days in the year that a given village experiences wind in their direction, from a coal plant. Mean wind in this new specification is 0.066, indicating that a village in the sample experiences wind in its direction on around 6.6 percent of days, or 24 days of the year.

We see a strong, negative relationship between pollution (lagged wind) and nightlights. Going from the 25th percentile to the 75th percentile of wind exposure is synonymous with a decrease in the growth of nightlights within a village by around 0.7 percent. While this is a relatively small magnitude, it is worth noting that there are more than 100,000 villages in this sample. While the effect for a single village might be small, the overall, aggregate effect on growth is quite substantial.

\hypertarget{conclusion}{%
\section{Conclusion}\label{conclusion}}

\FloatBarrier
\newpage
\singlespacing

\hypertarget{references}{%
\section*{References}\label{references}}
\addcontentsline{toc}{section}{References}

\hypertarget{refs}{}
\begin{CSLReferences}{1}{0}
\leavevmode\vadjust pre{\hypertarget{ref-aragon2016polluting}{}}%
Aragón, Fernando M, and Juan Pablo Rud. 2016. {``{Polluting industries and agricultural productivity: Evidence from mining in Ghana}.''} \emph{{The Economic Journal}} 126 (597): 1980--2011.

\leavevmode\vadjust pre{\hypertarget{ref-arceo2016does}{}}%
Arceo, Eva, Rema Hanna, and Paulina Oliva. 2016. {``{Does the effect of pollution on infant mortality differ between developing and developed countries? Evidence from Mexico City}.''} \emph{{The Economic Journal}} 126 (591): 257--80.

\leavevmode\vadjust pre{\hypertarget{ref-almn2021}{}}%
Asher, Sam, Tobias Lunt, Ryu Matsuura, and Paul Novosad. 2021. {``{Development Research at High Geographic Resolution: An Analysis of Night Lights, Firms, and Poverty in India using the SHRUG Open Data Platform}.''} \emph{{The World Bank Economic Review}}.

\leavevmode\vadjust pre{\hypertarget{ref-bencko1977health}{}}%
Bencko, Vladimir, and Karel Symon. 1977. {``Health Aspects of Burning Coal with a High Arsenic Content.''} \emph{{Environmental Research}} 13 (3): 378--85.

\leavevmode\vadjust pre{\hypertarget{ref-brunekreef2002air}{}}%
Brunekreef, Bert, and Stephen T Holgate. 2002. {``{Air pollution and health}.''} \emph{{The Lancet}} 360 (9341): 1233--42.

\leavevmode\vadjust pre{\hypertarget{ref-burke2021using}{}}%
Burke, Marshall, Anne Driscoll, David B Lobell, and Stefano Ermon. 2021. {``{Using satellite imagery to understand and promote sustainable development}.''} \emph{Science} 371 (6535): eabe8628.

\leavevmode\vadjust pre{\hypertarget{ref-chang2019effect}{}}%
Chang, Tom Y, Joshua Graff Zivin, Tal Gross, and Matthew Neidell. 2019. {``{The effect of pollution on worker productivity: evidence from call center workers in China}.''} \emph{{American Economic Journal: Applied Economics}} 11 (1): 151--72.

\leavevmode\vadjust pre{\hypertarget{ref-chen2022effect}{}}%
Chen, Shuai, Paulina Oliva, and Peng Zhang. 2022. {``{The effect of air pollution on migration: evidence from China}.''} \emph{{Journal of Development Economics}} 156: 102833.

\leavevmode\vadjust pre{\hypertarget{ref-currie2014we}{}}%
Currie, Janet, Joshua Graff Zivin, Jamie Mullins, and Matthew Neidell. 2014. {``{What do we know about short-and long-term effects of early-life exposure to pollution?}''} \emph{{Annual Review Resource Economics}} 6 (1): 217--47.

\leavevmode\vadjust pre{\hypertarget{ref-ebenstein2016long}{}}%
Ebenstein, Avraham, Victor Lavy, and Sefi Roth. 2016. {``{The long-run economic consequences of high-stakes examinations: Evidence from transitory variation in pollution}.''} \emph{{American Economic Journal: Applied Economics}} 8 (4): 36--65.

\leavevmode\vadjust pre{\hypertarget{ref-gangopadhyay2022new}{}}%
Gangopadhyay, Prasun K, Paresh B Shirsath, Vinay K Dadhwal, and Pramod K Aggarwal. 2022. {``{A new two-decade (2001--2019) high-resolution agricultural primary productivity dataset for India}.''} \emph{{Scientific Data}} 9 (1): 1--12.

\leavevmode\vadjust pre{\hypertarget{ref-gibson2021night}{}}%
Gibson, John, Susan Olivia, Geua Boe-Gibson, and Chao Li. 2021. {``{Which night lights data should we use in economics, and where?}''} \emph{{Journal of Development Economics}} 149: 102602.

\leavevmode\vadjust pre{\hypertarget{ref-graff2012impact}{}}%
Graff Zivin, Joshua, and Matthew Neidell. 2012. {``{The impact of pollution on worker productivity}.''} \emph{{American Economic Review}} 102 (7): 3652--73.

\leavevmode\vadjust pre{\hypertarget{ref-hammer2020global}{}}%
Hammer, Melanie S, Aaron van Donkelaar, Chi Li, Alexei Lyapustin, Andrew M Sayer, N Christina Hsu, Robert C Levy, et al. 2020. {``Global Estimates and Long-Term Trends of Fine Particulate Matter Concentrations (1998--2018).''} \emph{{Environmental Science \& Technology}} 54 (13): 7879--90.

\leavevmode\vadjust pre{\hypertarget{ref-hanna2015effect}{}}%
Hanna, Rema, and Paulina Oliva. 2015. {``{The effect of pollution on labor supply: Evidence from a natural experiment in Mexico City}.''} \emph{{Journal of Public Economics}} 122: 68--79.

\leavevmode\vadjust pre{\hypertarget{ref-he2019severe}{}}%
He, Jiaxiu, Haoming Liu, and Alberto Salvo. 2019. {``{Severe air pollution and labor productivity: Evidence from industrial towns in China}.''} \emph{{American Economic Journal: Applied Economics}} 11 (1): 173--201.

\leavevmode\vadjust pre{\hypertarget{ref-heck1982assessment}{}}%
Heck, Walter W, OC Taylor, Richard Adams, Gail Bingham, Joseph Miller, Eric Preston, and Leonard Weinstein. 1982. {``{Assessment of crop loss from ozone}.''} \emph{{Journal of the Air Pollution Control Association}} 32 (4): 353--61.

\leavevmode\vadjust pre{\hypertarget{ref-heft2018robust}{}}%
Heft-Neal, Sam, Jennifer Burney, Eran Bendavid, and Marshall Burke. 2018. {``{Robust relationship between air quality and infant mortality in Africa}.''} \emph{Nature} 559 (7713): 254--58.

\leavevmode\vadjust pre{\hypertarget{ref-heft2019air}{}}%
Heft-Neal, Sam, Jennifer Burney, Eran Bendavid, Kara Voss, and Marshall Burke. 2019. {``{Air pollution and infant mortality: Evidence from Saharan dust}.''} {National Bureau of Economic Research}.

\leavevmode\vadjust pre{\hypertarget{ref-henderson2012measuring}{}}%
Henderson, J Vernon, Adam Storeygard, and David N Weil. 2012. {``Measuring Economic Growth from Outer Space.''} \emph{{American Economic Review}} 102 (2): 994--1028.

\leavevmode\vadjust pre{\hypertarget{ref-kampa2008human}{}}%
Kampa, Marilena, and Elias Castanas. 2008. {``{Human health effects of air pollution}.''} \emph{{Environmental Pollution}} 151 (2): 362--67.

\leavevmode\vadjust pre{\hypertarget{ref-li2014health}{}}%
Li, Ya-Ru, and Jacqueline MacDonald Gibson. 2014. {``{Health and air quality benefits of policies to reduce coal-fired power plant emissions: a case study in North Carolina}.''} \emph{{Environmental Science \& Technology}} 48 (17): 10019--27.

\leavevmode\vadjust pre{\hypertarget{ref-marshall1997hidden}{}}%
Marshall, Fiona, Mike Ashmore, Fiona Hinchcliffe, et al. 1997. \emph{{A hidden threat to food production: Air pollution and agriculture in the developing world}}. {International Institute for Environment and Development.}

\leavevmode\vadjust pre{\hypertarget{ref-pope2006health}{}}%
Pope III, C Arden, and Douglas W Dockery. 2006. {``{Health effects of fine particulate air pollution: lines that connect}.''} \emph{{Journal of the Air \& Waste Management Association}} 56 (6): 709--42.

\leavevmode\vadjust pre{\hypertarget{ref-wen2022lower}{}}%
Wen, Jeff, and Marshall Burke. 2022. {``{Lower test scores from wildfire smoke exposure}.''} \emph{{Nature Sustainability}} 5 (11): 947--55.

\end{CSLReferences}

\FloatBarrier
\newpage

\hypertarget{appendix-a}{%
\section*{Appendix A}\label{appendix-a}}
\addcontentsline{toc}{section}{Appendix A}

\setcounter{table}{0} \renewcommand{\thetable}{A\arabic{table}} \setcounter{figure}{0} \renewcommand{\thefigure}{A\arabic{figure}} 
\FloatBarrier

\begin{table}[H]

\caption{\label{tab:data}Remote sensing data sources}
\centering
\begin{threeparttable}
\begin{tabular}[t]{>{\raggedright\arraybackslash}p{2cm}>{\centering\arraybackslash}p{4.5cm}>{\centering\arraybackslash}p{3.5cm}>{\centering\arraybackslash}p{3.5cm}}
\toprule
  & source & geographic coverage & temporal coverage\\
\midrule
shapefile & Asher et al. (2021) & India & \\
coal plants & Global Energy Monitor & global & yearly\\
wind & NCAR & global & daily\\
pollution & Hammer et al. (2020) & global & monthly\\
agriculture & Angopadhyay et al. (2022) & global & two seasons/year\\
nightlights & Asher et al. (2021) & India & yearly\\
\bottomrule
\end{tabular}
\begin{tablenotes}
\item Note: NCAR stands for the National Center for Atmospheric Research. Additional information on Global Energy Monitor and NCAR can be found on their websites, at https://globalenergymonitor.org/projects/global-coal-plant-tracker and https://climatedataguide.ucar.edu/, respectively.
\end{tablenotes}
\end{threeparttable}
\end{table}

\newpage
\begin{table}[H]

\caption{\label{tab:yieldtablepostplant}Wind direction and agricultural productivity, post-coal plant only}
\centering
\begin{threeparttable}
\begin{tabular}[t]{>{\raggedright\arraybackslash}p{3cm}>{\centering\arraybackslash}p{2cm}>{\centering\arraybackslash}p{2cm}>{\centering\arraybackslash}p{2cm}>{\centering\arraybackslash}p{2cm}>{\centering\arraybackslash}p{2cm}}
\toprule
\multicolumn{1}{c}{ } & \multicolumn{3}{c}{all} & \multicolumn{1}{c}{monsoon} & \multicolumn{1}{c}{winter} \\
\cmidrule(l{3pt}r{3pt}){2-4} \cmidrule(l{3pt}r{3pt}){5-5} \cmidrule(l{3pt}r{3pt}){6-6}
  & (1) & (2) & (3) & (4) & (5)\\
\midrule
wind & -0.003*** & -0.002*** & -0.0007*** & -0.002*** & -0.002***\\
 & (0.0002) & (0.0002) & (9.7e-5) & (0.0002) & (0.0003)\\
rain (z) &  & 0.031*** & 0.010*** & 0.103*** & 0.016***\\
 &  & (0.0005) & (0.001) & (0.002) & (0.0005)\\
\textbf{fixed effects:} & \textbf{} & \textbf{} & \textbf{} & \textbf{} & \textbf{}\\
village-season & Yes & Yes & Yes & No & No\\
year & Yes & Yes & No & Yes & Yes\\
district-year-season & No & No & Yes & No & No\\
village & No & No & No & Yes & Yes\\
\midrule
observations & 2,062,036 & 2,046,104 & 2,046,104 & 1,089,013 & 957,091\\
\bottomrule
\end{tabular}
\begin{tablenotes}
\item Note: Standard errors are in parentheses and are clustered at the village level. The sample is restricted to village-years after a coal plant was built near the village.
\item * p<0.10 ** p<0.05 *** p<0.01
\end{tablenotes}
\end{threeparttable}
\end{table}

\FloatBarrier
\newpage
\begin{table}

\caption{\label{tab:yieldtableleads}Agricultural productivity and pollution leads}
\centering
\begin{threeparttable}
\begin{tabular}[t]{>{\raggedright\arraybackslash}p{3cm}>{\centering\arraybackslash}p{2cm}>{\centering\arraybackslash}p{2cm}}
\toprule
  & (1) & (2)\\
\midrule
wind (lag) & 0.004*** & 0.0006*\\
 & (0.0007) & (0.0003)\\
wind & -0.002*** & -0.0007***\\
 & (0.0004) & (0.0002)\\
wind (lead) & -0.004*** & -0.0001\\
 & (0.0001) & (9.8e-5)\\
rain (z) & 0.021*** & 0.008***\\
 & (0.0005) & (0.001)\\
\textbf{fixed effects:} & \textbf{} & \textbf{}\\
village-season & Yes & Yes\\
year & Yes & No\\
district-year-season & No & Yes\\
\midrule
observations & 1,979,635 & 1,979,635\\
\bottomrule
\end{tabular}
\begin{tablenotes}
\item Note: Standard errors are in parentheses and are clustered at the village level.
\item * p<0.10 ** p<0.05 *** p<0.01
\end{tablenotes}
\end{threeparttable}
\end{table}

\FloatBarrier
\newpage
\begin{figure}
\includegraphics{draft_files/figure-latex/laborplot-1} \caption[Timing of household surveys for the National Sample Survey]{Timing of household surveys for the National Sample Survey}\label{fig:laborplot}\floatfoot*{Note: The histogram shows the distribution of survey dates for the four waves of the National Sample Survey (NSS) for the five waves used in this paper.}
\end{figure}

\FloatBarrier
\newpage
\begin{table}

\caption{\label{tab:labortablemonsoon}Wind direction and labor allocation, monsoon season only}
\centering
\begin{threeparttable}
\begin{tabular}[t]{>{\raggedright\arraybackslash}p{3cm}>{\centering\arraybackslash}p{1.5cm}>{\centering\arraybackslash}p{1.5cm}>{\centering\arraybackslash}p{1.5cm}>{\centering\arraybackslash}p{1.5cm}>{\centering\arraybackslash}p{1.5cm}>{\centering\arraybackslash}p{1.5cm}}
\toprule
  & all & all & self & wage & farm & non-farm\\
\midrule
wind & -0.021 & -0.033 & 0.038 & -0.071** & 0.072** & -0.105**\\
 & (0.028) & (0.029) & (0.034) & (0.033) & (0.036) & (0.046)\\
controls & No & Yes & Yes & Yes & Yes & Yes\\
\textbf{fixed effects:} & \textbf{} & \textbf{} & \textbf{} & \textbf{} & \textbf{} & \textbf{}\\
district & Yes & Yes & Yes & Yes & Yes & Yes\\
year & Yes & Yes & Yes & Yes & Yes & Yes\\
\textbf{varying slopes:} & \textbf{} & \textbf{} & \textbf{} & \textbf{} & \textbf{} & \textbf{}\\
year (by district) & Yes & Yes & Yes & Yes & Yes & Yes\\
\midrule
observations & 359,652 & 359,576 & 359,576 & 359,576 & 359,576 & 359,576\\
\bottomrule
\end{tabular}
\begin{tablenotes}
\item Note: Standard errors are in parentheses and are clustered at the district level. Control variables include female, age, age squared, and (years of) education.
\item * p<0.10 ** p<0.05 *** p<0.01
\end{tablenotes}
\end{threeparttable}
\end{table}

\FloatBarrier
\newpage
\begin{table}

\caption{\label{tab:labortablewinter}Wind direction and labor allocation, winter season only}
\centering
\begin{threeparttable}
\begin{tabular}[t]{>{\raggedright\arraybackslash}p{3cm}>{\centering\arraybackslash}p{1.5cm}>{\centering\arraybackslash}p{1.5cm}>{\centering\arraybackslash}p{1.5cm}>{\centering\arraybackslash}p{1.5cm}>{\centering\arraybackslash}p{1.5cm}>{\centering\arraybackslash}p{1.5cm}}
\toprule
  & all & all & self & wage & farm & non-farm\\
\midrule
wind & 0.003 & 0.002 & 0.011 & -0.009 & 0.024 & -0.022\\
 & (0.029) & (0.026) & (0.028) & (0.025) & (0.034) & (0.039)\\
controls & No & Yes & Yes & Yes & Yes & Yes\\
\textbf{fixed effects:} & \textbf{} & \textbf{} & \textbf{} & \textbf{} & \textbf{} & \textbf{}\\
district & Yes & Yes & Yes & Yes & Yes & Yes\\
year & Yes & Yes & Yes & Yes & Yes & Yes\\
\textbf{varying slopes:} & \textbf{} & \textbf{} & \textbf{} & \textbf{} & \textbf{} & \textbf{}\\
year (by district) & Yes & Yes & Yes & Yes & Yes & Yes\\
\midrule
observations & 375,956 & 375,883 & 375,883 & 375,883 & 375,883 & 375,883\\
\bottomrule
\end{tabular}
\begin{tablenotes}
\item Note: Standard errors are in parentheses and are clustered at the district level. Control variables include female, age, age squared, and (years of) education.
\item * p<0.10 ** p<0.05 *** p<0.01
\end{tablenotes}
\end{threeparttable}
\end{table}

\end{document}
